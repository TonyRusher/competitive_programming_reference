% compile this way
% pdflatex -shell-escape reference.tex  

\documentclass[10pt]{article}
\usepackage[utf8x]{inputenc}
\usepackage[spanish]{babel}
\decimalpoint
\usepackage{amsmath}
\usepackage{amsthm}
\usepackage{amssymb}
\usepackage{graphicx}
\usepackage[margin=0.6in,landscape]{geometry}
\usepackage{fancyhdr}
\usepackage[inline]{enumitem}
\usepackage{float}
\usepackage{cancel}
\usepackage{bigints}
\usepackage{listings}
\usepackage{xcolor}
\usepackage{listingsutf8}
\usepackage{algpseudocode}
\usepackage{algorithm}
\usepackage{apacite}
\usepackage{minted}
\usepackage{tcolorbox}
\usepackage{multicol}
\usepackage{hyperref}
\hypersetup{
	colorlinks,
	citecolor=black,
	filecolor=black,
	linkcolor=black,
	urlcolor=black
}
\twocolumn
\usepackage{tipa}
\pagestyle{fancy}
\newcommand{\xvdash}[1]{%
	\vdash^{\mkern-10mu\scriptscriptstyle\rule[-.9ex]{0pt}{0pt}#1}%
}
\setlength{\headheight}{15pt} 
\lhead{Reference}
\rhead{\thepage}
\lfoot{ESCOM-IPN}
\rfoot{Tony Rusher}
\renewcommand{\footrulewidth}{0.5pt}
\setlength{\parskip}{0.5em}
\newcommand{\ve}[1]{\overrightarrow{#1}}
\newcommand{\abs}[1]{\left\lvert #1 \right\lvert}
\newcommand{\blank}{\text{\textcrb}}
\title{Reference}

\lstdefinestyle{customc}{
	belowcaptionskip=1\baselineskip,
	breaklines=true,
	frame=L,
	xleftmargin=\parindent,
	language=C++,
	showstringspaces=false,
	basicstyle=\ttfamily,
	keywordstyle=\bfseries\color{green!40!black},
	commentstyle=\itshape\color{purple!40!black},
	identifierstyle=\color{blue},
	numbers=left,
	stringstyle=\color{orange},
}
\newcommand{\genstirlingI}[3]{%
	\genfrac{[}{]}{0pt}{#1}{#2}{#3}%
}
\newcommand{\genstirlingII}[3]{%
	\genfrac{\{}{\}}{0pt}{#1}{#2}{#3}%
}
\newcommand{\genEuler}[3]{%
	\genfrac{<}{>}{0pt}{#1}{#2}{#3}%
}
\newcommand{\stirlingI}[2]{\genstirlingI{}{#1}{#2}}
\newcommand{\stirlingII}[2]{\genstirlingII{}{#1}{#2}}
\newcommand{\euler}[2]{\genEuler{}{#1}{#2}}

\begin{document}
	% \begin{multicols*}{2}
			\section{Template}
			\inputminted[tabsize=2,breaklines,fontsize=\small]{c++}{template.cpp}
			\section{Data Structure}
				\subsection{Segment tree with lazy}
					\inputminted[tabsize=2,breaklines,fontsize=\small]{c++}{DataStructures/SegmentTree.cpp}
				\subsection{Segment tree Geometric sum}	
					\inputminted[tabsize=2,breaklines,fontsize=\small]{c++}{DataStructures/ST-geometricSum.cpp}
				\subsection{Treap }	
					\inputminted[tabsize=2,breaklines,fontsize=\small]{c++}{DataStructures/treap.cpp}
				\subsection{Implicit Treap}
					\inputminted[tabsize=2,breaklines,fontsize=\small]{c++}{DataStructures/implicitTreap.cpp}
				\subsection{Policy base }	
					\inputminted[tabsize=2,breaklines,fontsize=\small]{c++}{DataStructures/policyBase.cpp}
				\subsection{SQRT and MO's}
					\inputminted[tabsize=2,breaklines,fontsize=\small]{c++}{DataStructures/sqrtDecom.cpp}
				\subsection{Convex Hull Trick}
					\inputminted[tabsize=2,breaklines,fontsize=\small]{c++}{DataStructures/ConvexHullTrick.cpp}
				\subsection{BIT}
					\inputminted[tabsize=2,breaklines,fontsize=\small]{c++}{DataStructures/BIT.cpp}
				\subsection{merge sort tree}
					\inputminted[tabsize=2,breaklines,fontsize=\small]{c++}{DataStructures/mergeSortTree.cpp}
			\section{Binary Search}
				\inputminted[tabsize=2,breaklines,fontsize=\small]{c++}{algorithms/binarySearch.cpp}
				\inputminted[tabsize=2,breaklines,fontsize=\small]{c++}{algorithms/ternarySearch.cpp}
			\section{Flujos}
				\subsubsection{Dinic}
					\inputminted[tabsize=2,breaklines,fontsize=\small]{c++}{Flows/Dinic.cpp}
				\subsubsection{Ford-Fulkerson}
					\inputminted[tabsize=2,breaklines,fontsize=\small]{c++}{Flows/Ford-Fulkerson.cpp}
				\subsubsection{maxflow mincost}
					\inputminted[tabsize=2,breaklines,fontsize=\small]{c++}{Flows/mincostMaxflow.cpp}
			\section{Strings}
				\subsection{aho-corasik}
					\inputminted[tabsize=2,breaklines,fontsize=\small]{c++}{Strings/Aho-corasik.cpp}
			\section{Math}
				\subsection{FFT}
					\inputminted[tabsize=2,breaklines,fontsize=\small]{c++}{FFT/FFT.cpp}
				\subsection{NTT}
					\inputminted[tabsize=2,breaklines,fontsize=\small]{c++}{FFT/NTT.cpp}
				\subsection{Field extension}
					\inputminted[tabsize=2,breaklines,fontsize=\small]{c++}{DataStructures/EX.cpp}
				\subsection{nCr}
					\inputminted[tabsize=2,breaklines,fontsize=\small]{c++}{Math/nCr.cpp}
				\subsection{Discrete Root}
					\inputminted[tabsize=2,breaklines,fontsize=\small]{c++}{Math/discreteRoot.cpp}
				\subsection{Miller Rabin}
					\inputminted[tabsize=2,breaklines,fontsize=\small]{c++}{Math/MillerRabin.cpp}
			\section{Graphs}
				\subsection{bfs-dfs}
					\inputminted[tabsize=2,breaklines,fontsize=\small]{c++}{Graph/bfs-dfs.cpp}
				\subsection{cicles}
					\inputminted[tabsize=2,breaklines,fontsize=\small]{c++}{Graph/cicleAndBipartite.cpp}
				\subsection{Dijkstra}
					\inputminted[tabsize=2,breaklines,fontsize=\small]{c++}{Graph/dijkstra.cpp}
				\subsection{DSU}
					\inputminted[tabsize=2,breaklines,fontsize=\small]{c++}{Graph/DSU.cpp}
				\subsection{FloydWarshall}
					\inputminted[tabsize=2,breaklines,fontsize=\small]{c++}{Graph/floydWarshall.cpp}
				\subsection{Kruskal}
					\inputminted[tabsize=2,breaklines,fontsize=\small]{c++}{Graph/Kruskal.cpp}
				\subsection{topoSort}
					\inputminted[tabsize=2,breaklines,fontsize=\small]{c++}{Graph/topoSort.cpp}
			\section{Tree}
				\inputminted[tabsize=2,breaklines,fontsize=\small]{c++}{Graph/Tree.cpp}
				\subsection{HLD}
					\inputminted[tabsize=2,breaklines,fontsize=\small]{c++}{Graph/HLD.cpp}
			\section{Memorización}
				\subsection{Knapsack}
					\inputminted[tabsize=2,breaklines,fontsize=\small]{c++}{Memorizacion/Knapsack.cpp}
		
		
	% \end{multicols*}
	
\end{document}